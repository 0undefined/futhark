\documentclass[oneside]{memoir}
\usepackage{alltt}
\usepackage{amsmath}
\usepackage[capitalise,noabbrev]{cleveref}
\usepackage{fancyvrb}
\usepackage{color}

% Remove chapter numbering, as we only use one chapter in this
% document.
\counterwithout{section}{chapter}

\newcommand\boolt[0]{\texttt{bool}}
\newcommand\realt[0]{\texttt{real}}
\newcommand\chart[0]{\texttt{char}}
\newcommand\intt[0]{\texttt{int}}
\newcommand\aliases[1]{\textrm{aliases}(#1)}
\renewcommand\tilde[0]{{\raise.17ex\hbox{$\scriptstyle\sim$}}}
\newcommand{\LO}{$\mathcal{L}_0$}

% The colorcode environment lets us use colour markup inside verbatim blocks.
\DefineVerbatimEnvironment{colorcode}%
        {Verbatim}{fontsize=\small,commandchars=\\\{\}}

\definecolor{DikuRed}{RGB}{130,50,32}
\newcommand{\emp}[1]{\textcolor{DikuRed}{ #1}}
\definecolor{CosGreen}{RGB}{10,100,70}
\newcommand{\emphh}[1]{\textcolor{CosGreen}{ #1}}

\title{The \LO{} Programming Language}
\date{\today}
\author{Troels Henriksen (athas@sigkill.dk)}

\begin{document}

\maketitle

\section{Introduction}

This document has been prepared as an informal overview of the L0
language.  In time, we hope to develop it into a formal specification,
with accompanying proofs of correctness, but for now, words will have
to suffice where formulae would be ideal.

L0 is an eagerly evaluated, purely functional language with built-in
arrays and second-order array combinators, which is intended for use
as an \emph{intermediate language} (sometimes called a \emph{core
  language}) with a focus towards efficient execution on vector
hardware (GPUs).  While L0 is not designed to be a pleasant
programming experience for humans, it does have a textual syntax to
facilitate the writing of benchmark programs.

Section \ref{sec:l0-overview} gives a cursory tour of the L0 language.

Section \ref{sec:l0-reference} proceeds with a syntax-oriented
description of every language construct in L0.

Section \ref{sec:l0-uniqueness-types} describes the \emph{uniqueness
  types} of L0; a type system feature that permits in-place
modification of arrays as long as it can be proven that referential
transparency is upheld.

\section{Language overview}
\label{sec:l0-overview}

The \LO{} programming language is a purely functional, call-by-value,
mostly first-order language that permits bulk operations on arrays
using \textit{second-order array combinators} (SOACs).

The primary idea behind \LO{} is to design a language that has enough
expressive power to conveniently express complex programs, yet is also
amenable to aggressive optimisation and parallelisation.
Unfortunately, as the expressive power of a language grows, the
difficulty of optimisation often rises likewise.  For example, we
support nested parallelism, despite the complexities of efficiently
mapping to the flat parallelism supported by hardware, as a great many
programs depend on this feature.  On the other hand, we do not support
non-regular arrays, as they complicate size analysis a great deal.
The fact that \LO{} is purely functional is intended to give an
optimising compiler more leeway in rearranging the code and performing
high-level optimisations.  It is also the plan to eventually design a
rigorous cost model for \LO{}, although this work has not yet been
completed.

\subsection{First-order \LO{}}
\label{sec:fo-l0}

\begin{figure}[bt]
\begin{tabular}{lrll}
$t$ & $::=$ & \texttt{int} & (Integers) \\
& $|$ & \texttt{real} & (Floats) \\
& $|$ & \texttt{bool} & (Booleans) \\
& $|$ & \texttt{char} & (Characters) \\
& $|$ & \texttt{\{$t_{1}$, \ldots, $t_{n}$\}} & (Tuples) \\
& $|$ & \texttt{[$t$]} & (Arrays) \\
& $|$ & \texttt{*[$t$]} & (Unique arrays) \\
\\
$k$ & $::=$ & $n$ & (Integer)\\
& $|$ & $x$ & (Decimal number) \\
& $|$ & $b$ & (Boolean)\\
& $|$ & $c$ & (Character)\\
& $|$ & \texttt{\{$v_{1},\ldots,v_{n}$\}} & (Tuple) \\
& $|$ & \texttt{[$v_{1},\ldots,v_{n}$]} & (Array) \\
\\
$p$ & $::=$ & \texttt{\textbf{id}} & (Name pattern)\\
& $|$ & \texttt{\{$p_{1}$, \ldots, $p_{n}$\}} & (Tuple pattern) \\
\end{tabular}
\caption{\LO{} syntax}
\label{fig:fo-syntax}
\end{figure}

\begin{figure}[bt]
\begin{tabular}{lrll}
$e$ & $::=$ & $k$ & (Constant)\\
& $|$ & $v$ & (Variable)\\
& $|$ & \texttt{\{$e_{1}$,\ldots,$e_{n}$\}} & (Tuple expression) \\
& $|$ & \texttt{[$e_{1}$,\ldots,$e_{n}$]} & (Array expression) \\
& $|$ & $e_{1} \odot{} e_{2}$ & (Binary operator) \\
& $|$ & \texttt{-$e$} & (Prefix minus) \\
& $|$ & \texttt{not $e$} & (Logical negation) \\
& $|$ & \texttt{if $e_{1}$ then $e_{2}$ else $e_{3}$} & (Branching) \\
& $|$ & \texttt{$v$[$e_{1}$, \ldots, $e_{n}$]} & (Indexing) \\
& $|$ & \texttt{$v$($e_{1}$, \ldots, $e_{n}$)} & (Function call) \\
& $|$ & \texttt{let $p$ = $e_{1}$ in $e_{2}$} & (Pattern binding) \\
& $|$ & \texttt{zip($e_{1}$, \ldots, $e_{n}$)} & (Zipping) \\
& $|$ & \texttt{unzip($e$)} & (Unzipping) \\
& $|$ & \texttt{iota($e$)} & (Range) \\
& $|$ & \texttt{replicate($e_{n}$, $e_{v}$)} & (Replication) \\
& $|$ & \texttt{size($e$)} & (Array length) \\
& $|$ & \texttt{reshape(($e_{1}$,\ldots,$e_{n}$), $e$)} & (Array reshape) \\
& $|$ & \texttt{transpose($e$)} & (Transposition) \\
& $|$ & \texttt{split($e_{1}$, $e_{2}$)} & (Split $e_{2}$ at index $e_{1}$) \\
& $|$ & \texttt{concat($e_{1}$, $e_{2}$)} & (Concatenation) \\
& $|$ & \texttt{let $v_{1}$ = $v_{2}$ with} & (In-place update) \\
&     & \texttt{\ \ [$e_{1}$,\ldots,$e_{n}$] <- $e_{v}$} \\
&     & \texttt{in $e_{b}$} \\
& $|$ & \texttt{loop ($p$ = $e_{1}$) =} & (Loop) \\
&     & \texttt{\ \ for $v$ < $e_{2}$ do $e_{3}$} \\
&     & \texttt{in $e_{4}$} \\
\end{tabular}
\\
\begin{tabular}{lrll}
$fun$ & $::=$ & \texttt{fun $t$ $v$($t_{1}$ $v_{1}$,\ldots $t_{n}$ $v_{n}$) = $e$} \\
\\
$prog$ & $::=$ & $\epsilon$ \\
       & $|$   & $fun$ $prog$
\end{tabular}
\caption{\LO{} syntax, continued}
\label{fig:fo-syntax-continued}
\end{figure}

The syntax of \LO{}, as seen on \cref{fig:fo-syntax} and
\cref{fig:fo-syntax-continued}, is heavily inspired by Haskell and
Standard ML.  An identifier starts with a letter, followed by any
number of letters, digits and underscores.  Numeric, string and
character literals use the same notation as Haskell (which is very
similar to C), including all escape characters.  Comments are
indicated with \texttt{//} and span to end of line.

An \LO{} program consists of a sequence of \emph{function
  definitions}, of the following form.

\begin{colorcode}
  fun \textit{return-type} \textit{name}(\textit{params...}) = \textit{body}
\end{colorcode}

A function must declare both its return type and the types of all its
parameters.  All functions (except for inline anonymous functions; see
below) are defined globally.  \LO{} does not use type inference.
Symbolic constants are not supported, although 0-ary functions can be
defined.  As a concrete example, here is the recursive definition of
the factorial function in \LO{}.
\begin{colorcode}
  fun int fact(int n) =
    if n = 0 then 1
             else n * fact(n-1)
\end{colorcode}
Indentation has no syntactical significance in \LO{}, but recommended for
readability.

The syntax for tuple types is a comma-separated list of types or
values enclosed in braces, so \texttt{\{int, real\}} is a pair of an
integer and a floating-point number.  Both single-element and empty
tuples are permitted.  Array types are written as the element type
surrounded by brackets, meaning that \texttt{[int]} is a
one-dimensional array of integers, and \texttt{[[[\{int, real\}]]]} is a
three-dimensional array of tuples of integers and floats.  An
immediate array is written as a sequence of elements enclosed by
brackets.
\begin{colorcode}
  [1, 2, 3]       // Array of type [int].
  [[1], [2], [3]] // Array of type [[int]].
\end{colorcode}
All arrays must be \emph{regular} (often termed \emph{full}) - for
example, all rows of a two-dimensional array must have the same number
of elements.
\begin{colorcode}
  [[1, 2], [3]]      // Compile-time error.
  [iota(1), iota(2)] // A run-time error if reached.
\end{colorcode}
The restriction to regular arrays simplifies size analysis and
optimisation.

Arrays are indexed using the common row-major notation, e.g.,
\texttt{a[i1, i2, i3...]}.  An indexing is said to be \textit{full} if
the number of given indexes is equal to the dimensionality of the
array.

A \texttt{let}-expression can be used to refer to the result of a
subexpression:
\begin{colorcode}
  let z = x + y in ...
\end{colorcode}
Recall that \LO{} is eagerly evaluated, so the right-hand side of the
\texttt{let} is evaluated exactly once, at the time it is first
encountered.

Two-way \texttt{if-then-else} is the only branching construct in \LO{}.
Pattern matching is supported in a limited way for taking apart
tuples, but this can only be done in \texttt{let}-bindings, and not
directly in a function argument list.  Specifically, the following
function definition is not valid.
\begin{colorcode}
  fun int sumpair(\{int, int\} \{x, y\}) = x + y // WRONG!
\end{colorcode}
Instead, we must use a let-binding explicitly, as follows.
\begin{colorcode}
  fun int sumpair(\{int, int\} t) =
    let \{x,y\} = t in x + y
\end{colorcode}
Pattern-matching in a binding is the only way to access the components
of a tuple.

Function calls are written as the function name followed by the
arguments enclosed in parentheses.  All function calls must be fully
saturated - currying is only permitted in SOACs (see
\cref{sec:soacs}).

\subsubsection{Sequential loops}
\label{sec:sequential-loops}

\LO{} has a built-in syntax for expressing certain tail-recursive
functions.  Consider the following tail-recursive formulation of a
function for computing the Fibonacci numbers.
\begin{colorcode}
  fun int fib(int n) = fibhelper(1,1,n)

  fun int fibhelper(int x, int y, int n) =
    if n = 1 then x else fibhelper(y, x+y, n-1)
\end{colorcode}
We can rewrite this using the \texttt{loop} construct.
\begin{colorcode}
  fun int fib(int n) =
    loop (\{x, y\} = \{1,1\}) = for i < n do
                              \{y, x+y\}
    in x
\end{colorcode}
The semantics of this is precisely as in the tail-recursive function
formulation.  In general, a loop
\begin{colorcode}
  loop (\emph{pat} = \emph{initial}) = for \emph{i} < \emph{bound} do \emph{loopbody}
  in \emph{body}
\end{colorcode}
has the following semantics:

\begin{enumerate}
  \item Bind \textit{pat} to the initial values given in \textit{initial}.
  \item While $\textit{i} < \textit{bound}$, evaluate
    \textit{loopbody}, rebinding \textit{pat} to be the value returned
    by the body.  At the end of each iteration, increment \textit{i}
    by one.
  \item Evaluate \textit{body} with \textit{pat} bound to its final
    value.
\end{enumerate}
Semantically, a \texttt{loop} expression is completely equivalent to a
call to its corresponding tail-recursive function.  For example,
denoting by \texttt{t} the type of \texttt{x}, the loop in
\cref{fig:loop-recursion} has the semantics of a call to the
tail-recursive function on the right-hand side.

The purpose of \texttt{loop} is partly to render some sequential
computations slightly more convenient, but primarily to express
certain very specific forms of recursive functions, specifically those
with a fixed iteration count.  This property can eventually be used
for analysis and optimisation, although the current \LO{} compiler
does not yet exploit this.

\begin{figure}
\begin{center}
\begin{minipage}{0.3\columnwidth}
\begin{colorcode}
loop (x = a) =
  for i < n do
    g(x)
in body
\end{colorcode}
\end{minipage}
\begin{minipage}{0.05\columnwidth}
$\Rightarrow$
\end{minipage}
\begin{minipage}{0.4\columnwidth}
\begin{colorcode}
fun t f(int i, int n, t x) =
  if i >= n then x
     else f(i+1, n, g(x))

let x = f(i, n, a)
in body
\end{colorcode}
\end{minipage}
\end{center}
\caption{Equivalence between loops and recursive functions}
\label{fig:loop-recursion}
\end{figure}

\subsubsection{In-place updates}
\label{sec:in-place-updates}

In an array-oriented programming language, a common task is to modify
some elements of an array.  In a pure language, we cannot permit free
mutation, but we can permit the creation of a duplicate array, where
some elements have been changed.  General modification of array
elements is done using the \texttt{let-with} construct.  In its most
general form, it looks as follows.
\begin{colorcode}
  let \textit{dest} = \textit{src} with [\textit{indexes}] <- \textit{value}
  in \textit{body}
\end{colorcode}
This evaluates \textit{body} with \textit{dest} bound to the value of
\textit{src}, except that the element(s) at the position given by
\textit{indexes} take on the new value \textit{value}.\footnote{Yes,
  this is the \emph{third} binding construct in the language, ignoring
  function abstraction!}  The given indexes need not be complete, but
in that case, \textit{value} must be an array of the proper size.  As
an example, here's how we could replace the third row of an $n\times3$
array.
\begin{colorcode}
  let b = a with [2] <- [1,2,3] in b
\end{colorcode}
Whenever $\textit{dest} = \textit{src}$, we can write
\begin{colorcode}
  let \textit{dest}[\textit{indexes}] = \textit{value} in \textit{body}
\end{colorcode}
as a shortcut.  Note that this has no special semantic meaning, but is
simply a case of normal name shadowing.

For example, the loop given below implements the ``imperative''
version of matrix multiplication of two $N\times N$ matrices.

\begin{colorcode}
fun *[[int]] matmultImp(int N, [[int]] a, [[int]] b) =
    let res = replicate(N, iota(N)) in
    loop (res) = for i < N do
        loop (res) = for j < N do
            let partsum =
                let res = 0 in
                loop (res) = for k < N do
                    let res = res + a[i,k] * b[k,j]
                    in  res
                in res
            in let res[i,j] = partsum in res
        in res
    in res
\end{colorcode}

With the naive implementation based on copying the source array,
executing the \texttt{let-with} expression would require memory
proportional to the entire source array, rather than proportional to
the slice we are changing.  This is not ideal.  Therefore, the
\texttt{let-with} construct has some unusual restrictions to permit
in-place modification of the \textit{src} array, as described in
\cref{sec:l0-uniqueness-types}.  Simply put, we track that
\textit{src} is never used again.  The consequence is that we can
guarantee that the execution of a \texttt{let-with} expression does
not involve any copying of the source array in order to create the
newly bound array, and therefore the time required for the update is
proportional to the section of the array we are updating, not the
entire array.  We can think of this as similar to array modification
in an imperative language.

\subsection{SOACs}
\label{sec:soacs}

The language presented in the previous section is in some sense
``sufficient'', in that it is Turing-complete, and can express
imperative-style loops in a natural way with \texttt{do}-loops.
However, \LO{} is not intended to be used in such a way - bulk
operations on arrays should be expressed via the four
\textit{second-order array combinators} (SOACs) shown in
\cref{fig:soacs}, as the optimisations covered in later chapters are
expressed as transformations on these.

\begin{figure}[bt]
\begin{tabular}{lrll}
$l$ & $::=$ & \texttt{fn $t$ ($t_{1}$ $v_{1}$, \ldots, $t_{n}$ $v_{n}$) => $e$} & (Anonymous function) \\
& $|$ & \texttt{\textbf{id} ($e_{1}$, \ldots, $e_{n}$)} & (Curried function) \\
& $|$ & \texttt{op $\odot$ ($e_{1}$, \ldots, $e_{n}$)} & (Curried operator) \\
\\
$e$ & $::=$ & \texttt{map($l$, $e$)} \\
    & $|$ & \texttt{filter($l$, $e$)} \\
    & $|$ & \texttt{reduce($l$, $x$, $e$)} \\
    & $|$ & \texttt{scan($l$, $x$, $e$)} \\
    & $|$ & \texttt{redomap($l_{r}$, $l_{m}$, $x$, $e$)} \\
\end{tabular}
\caption{Second-order array combinators}
\label{fig:soacs}
\end{figure}

The semantics of the \textsc{soac}s is identical to the similarly-named
higher-order functions found in many functional languages, but we
reproduce it here for completeness.  The types given are not \LO{}
types, but a Haskell-inspired notation, since the \textsc{soac}s cannot
be typed in \LO{} itself.
\begin{align*}
\texttt{map($f$,$a$)}
& :: (\alpha\rightarrow\beta)\rightarrow[\alpha]\rightarrow[\beta] \\
& \equiv \texttt{\{$f$($a$[0]), \ldots, $f$($a$[n])\}}
\end{align*}
\begin{align*}
\texttt{ filter}
& :: (\alpha\rightarrow\texttt{bool})\rightarrow[\alpha]\rightarrow[\alpha] \\
\texttt{ filter($f$,$a$)} & \equiv \texttt{\{$a$[i] | $f$($a$[i]) = \texttt{True }\}}
\end{align*}
\begin{align*}
\texttt{ reduce}
& :: (\alpha\rightarrow\alpha\rightarrow\alpha)\rightarrow\alpha\rightarrow[\alpha]\rightarrow\alpha \\
\texttt{ reduce($f$,$x$,$a$)} & \equiv \texttt{ $f$(\ldots($f$($f$($x$,$a$[0]), $a$[1])\ldots), $a$[n])}
\end{align*}
\begin{align*}
\texttt{ scan}
& :: (\alpha\rightarrow\alpha\rightarrow\alpha)\rightarrow\alpha\rightarrow[\alpha]\rightarrow[\alpha] \\
\texttt{ scan($f$,$x$,$a$)} & \equiv \texttt{\{$f$($x$,$a$[0]), $f$($f$($x$,$a$[0]),$a$[1]),\ldots\}}
\end{align*}

\texttt{scan} is an inclusive prefix scan, and returns an array of the
same outer size as the original array.  The functions given to
\texttt{reduce} and \texttt{scan} must be binary associative
operators, and the value given as the initial value of the accumulator
must be the neutral element for the function.  These properties are
not checked by the \LO{} compiler, and are the responsibility of the
programmer.

\texttt{redomap} is a special case -- it is not intended for use by
the programmer, but used internally for fusing \texttt{reduce} and
\texttt{map}.  Its semantics is as follows.
\begin{align*}
\texttt{redomap}
& :: (\alpha\rightarrow\alpha\rightarrow\alpha)\rightarrow(\alpha\rightarrow\beta\rightarrow\alpha)\\
& \quad\rightarrow\alpha\rightarrow[\beta]\rightarrow\alpha \\
\texttt{redomap($\odot$,$g$,$x$,$v$)} & \equiv \texttt{foldl($g$, $x$, $v$)}
\end{align*}
Note that the runtime semantics is a left-fold, not a normal \LO{}
\texttt{reduce}.  In particular, $g$ need not be associative.  We use
a Haskell-like syntax to explain the
rationale behind \texttt{redomap}:\\
$(\texttt{reduce}\ \odot\ e)\ \circ\ (\texttt{map}\ f)$ can be
formally transformed, via the list homomorphism (\textsc{lh})
promotion lemma, to an
equivalent form:  \\
$(\texttt{reduce}\ \odot\ e)\ \circ\ (\texttt{map}\ f)\
\emphh{\equiv}\ \texttt{reduce}\ \odot\
e\ \circ\ \texttt{map}\ (\emp{\texttt{reduce}\ \odot\ e\ \circ\ \texttt{map}\ f})\ \circ\ \texttt{split}_{p}$\\
where the original list is distributed to $p$ parallel processors,
each of which execute the original map-reduce computation sequentially
and, at the end, reduce in parallel the per-processor result using the
operator $\odot$.  Hence, the \textit{inner} map-reduce can be
rewritten as a left-fold:\\
$(\texttt{reduce}\ \odot\ e)\ \circ\ (\texttt{map}\ f)\
\emphh{\equiv}\ \texttt{reduce}\ \odot\
e\ \circ \texttt{map} \ (\emp{\texttt{foldl}\ g\ e})\ \circ\ \texttt{split}_{p}$\\
Where $g$ is a function generated from the composition of $f$ and
$\odot$.  It follows that in order to generate parallel code for \\
\texttt{(reduce $\odot$ $e$) $\circ$ (map $f$)} it is sufficient to
record either $\odot$ and $f$, or $\odot$ and $g$. We choose the
latter, i.e., \texttt{redomap($\odot$, $g$, $e$)}, because it allows a
richer compositional algebra for fusion.  In particular, it allows us
to fuse \texttt{reduce~$\circ$~map~$\circ$~filter} into a
\texttt{redomap} without duplicating computation.

\section{Language reference}
\label{sec:l0-reference}

The builtin types in \LO{} are \intt{}, \realt{}, \boolt{} and \chart{}, as
well as their combination in tuples and arrays.

The following list describes every syntactical language construct in
the language.

\begin{description}
  \item[\textit{constant}]\hfill\\
    Evaluates to itself.

  \item[\textit{var}]\hfill\\
    Evaluates to its value in the environment.

  \item[\texttt{\textit{x} \textbf{arithop} \textit{y}}] \hfill\\
    Evaluate the binary operator on its operands, which must both be
    of either type \intt{} or \realt.  The following operators are
    supported: \texttt{+}, \texttt{*}, \texttt{-}, \texttt{/},
    \texttt{\%}, \texttt{=}, \texttt{<}, \texttt{<=}, \texttt{pow}.

  \item[\texttt{\textit{x} \textbf{bitop} \textit{y}}] \hfill\\
    Evaluate the binary operator on its operands, which must both be
    of type \intt.  The following operators are supported:
    \texttt{\^}, \texttt{\&}, \texttt{|}, \texttt{$>>$}, \texttt{$<<$},
    i.e., bitwise xor, and, or, and arithmetic shift right and left.

  \item[\texttt{\textit{x} \&\& \textit{y}}]\hfill\\
    Logical conjunction; both operands must be of type \boolt.  Not
    short-circuiting, as this complicates program transformation.  If
    short-circuiting behaviour is desired, the programmer can use
    \texttt{if} explicitly.

  \item[\texttt{\textit{x} || \textit{y}}]\hfill\\
    Logical disjunction; both operands must be of type \boolt.  As
    with \texttt{\&\&}, not short-circuiting.

  \item[\texttt{not \textit{x}}]\hfill\\
    Logical negation of \textit{x}, which must be of type \boolt.

  \item[\texttt{-\textit{x}}]\hfill\\
    Numerical negation of \textit{x}, which must be of type \realt{} or \intt.

  \item[\texttt{a[i]}]\hfill\\
    Return the element at the given position in the array.  The index
    may be a comma-separated list of indexes.

  \item[\texttt{zip($a_1, \ldots, a_n$)}]\hfill\\
    Zips together the elements of the outer dimensions of arrays $a_1,
    \ldots, a_n$.  Static or runtime check is required to check that
    the sizes of the outermost dimension of arrays $a_1, \ldots, a_n$
    are the same.  If this invariant does not hold, program execution
    stops with an error.

  \item[\texttt{unzip($a$)}]\hfill\\
    If the type of $a$ is $[\{t_1, \ldots, t_n\}]$, the result is a
    tuple of $n$ arrays, i.e., $\{[t_1], \ldots, [t_n]\}$, otherwise
    it is a type error.

  \item[\texttt{iota(\textit{n})}]\hfill\\
    An array of the integers from $0$ to \textit{n}.

  \item[\texttt{replicate(\textit{n}, \textit{a})}]\hfill\\
    An array consisting of \textit{n} copies of \textit{a}.

  \item[\texttt{size($k$, \textit{a})}]\hfill\\
    The size of dimension $k$ of array \textit{a}.  $k$ must be a
    static integral constant.

  \item[\texttt{split(\textit{n}, \textit{a})}]\hfill\\
    Partitions the given array into two disjoint arrays
    \texttt{\textit{a}[$0\ldots{}n$]},
    \texttt{\textit{a}[$n+1\ldots{}$]}, returned as a tuple.

  \item[\texttt{concat(\textit{a}, \textit{b})}]\hfill\\
    Concatenate the rows/elements of one array with another.  The
    shape of the two arrays must be identical in all but the first
    dimension.

  \item[\texttt{copy(\textit{x})}]\hfill\\
    Return a deep copy of the argument.  Semantically, this is just
    the identity function, but it has special semantics related to
    uniqueness types as described in \cref{sec:l0-uniqueness-types}.

  \item[\texttt{reshape((\textit{dim}$_{1}$, \ldots, \textit{dim}$_{n}$), a)}]\hfill\\
    Reshape the elements of the given array into the specified shape.
    The number of elements in \textit{a} must be equal to
    $\texttt{dim}_{1}\times\ldots\times\texttt{dim}_{n}$.

  \item[\texttt{transpose(a)}]\hfill\\
    Return the transpose of \textit{a}.

  \item[\texttt{transpose(k,n,a)}]\hfill\\
    Return the generalised transpose of \textit{a}.  If
    \texttt{$b$=transpose($k$,$n$,$a$)}, then
    \[
    \texttt{a}[i_1, \ldots, i_k, i_{k+1}, \ldots, i_{k+n}, \ldots, i_q ]
    =
    \texttt{b}[i_1 , \ldots, i_{k+1} , \ldots, i_{k+n}, i_k, \ldots, i_q ].
    \]
    We will call this an operation an \textit{$(k,n)$-transposition}.
    Note that \texttt{transpose(0,1,a)} is the common two-dimensional
    transpose.

    Be aware that $k$ and $n$ must be static integer literals, and
    $k+n$ must be non-negative and smaller than the rank of
    \texttt{a}, or it is considered a type error.

  \item[\texttt{let \textit{pat} = \textit{e} in \textit{body}}]\hfill\\
    While evaluating \textit{body}, bind the names mentioned in
    \textit{pat} to the components in the corresponding positions of
    the value of \textit{e}.  We will refer to the expression
    \textit{e} as the ``right-hand side'' (or RHS).

  \item[\texttt{let \textit{dest} = \textit{src} with [\textit{index}] <- \textit{v} in \textit{body}}] \hfill \\
    Evaluate \textit{body} with \textit{dest} bound to the value of
    \textit{src}, except that the element(s) at the position given by
    the index take on the value of \textit{v}.  The given index need
    not be complete, but in that case, the value of \textit{v} must be
    an array of the proper size.

  \item[\texttt{if \textit{c} then \textit{a} else \textit{b}}]\hfill\\
    If \textit{c} evaluates to \textit{True}, evaluate \textit{a},
    else evaluate \textit{b}.

  \item[\texttt{loop (\textit{pat} = \textit{initial}) = for \textit{i} < \textit{bound} do \textit{loopbody} in \textit{body}}]\hfill
    \begin{enumerate}
    \item Bind \textit{pat} to the initial values given in \textit{initial}.
    \item While $\textit{i} < \textit{bound}$, evaluate \textit{loopbody},
      rebinding \textit{pat} to be the value returned by the body.
    \item Evaluate \textit{body} with \textit{pat} bound to its final
      value.
    \end{enumerate}

  \item[\texttt{map(\textit{f}, \textit{a})}]\hfill\\
    Apply \textit{f} to every element of \textit{a} and return the resulting array.

  \item[\texttt{reduce(\textit{f}, \textit{x}, \textit{a})}]\hfill\\
    Left-reduction with \textit{f} across the elements of \textit{a},
    with \textit{x} as the neutral element for \textit{f}.  \textit{f}
    must be associative, as the evaluation order is not otherwise
    specified.

  \item[\texttt{scan(\textit{f}, \textit{x}, \textit{a})}]\hfill\\
    Inclusive prefix-scan.

  \item[\texttt{filter(\textit{f}, \textit{a})}]\hfill\\
    Remove all those elements of \textit{a} that do not satisfy the
    predicate \textit{f}.

\end{description}

\subsection{Tuple shimming}

In a SOAC, if the given function expects $n$ arguments of types
$t_{1}, \ldots, t_{n}$, but the SOAC will call the function with a
single argument of type \texttt{\{$t_{1}, \ldots, t_{n}$\}} (that is,
a tuple), the \LO{} compiler will automatically generate an anonymous
unwrapping function.  This allows the following expression to
type-check (and run):

\begin{colorcode}
  map(op +, zip(as, bs))
\end{colorcode}

Without the tuple shimming, the above would cause an error, as
\texttt{op +} is a function that takes two arguments, but is passed a
two-element tuple by \texttt{map}.

\subsection{Arrays of tuples}

For reasons related to fusion, arrays of tuples are in a sense merely
syntactic sugar for tuples of arrays.  The type \texttt{[\{int,
  real\}]} is transformed to \texttt{\{[int], [real]\}} during the
compilation process, and all code interacting with arrays of tuples is
likewise transformed.  In most cases, this is fully transparent to the
programmer, but there are edge cases where the transformation is not
trivially an isomorphism.

Consider the type \texttt{[\{[int], [real]\}]}, which is transformed
into \texttt{\{[[int]], [[real]]\}}.  These two types are not
isomorphic, as the latter has more stringent demands as to the
fullness of arrays.  For example,
\begin{colorcode}
[
 \{[1],   [1.0]\},
 \{[2,3], [2.0]\}
]
\end{colorcode}
is a value of the former, but the first element of the
corresponding transformed tuple
\begin{colorcode}
\{
 [[1],   [2, 3]],
 [[1.0], [2.0]]
\}
\end{colorcode}
is not a full array.  Hence, when determining whether a program
generates full arrays, we must hence look at the \textit{transformed}
values - in a sense, the fullness requirement ``transcends'' the
tuples.

\section{Uniqueness Types}
\label{sec:l0-uniqueness-types}

While \LO{} is through and through a pure functional language, it may
occasionally prove useful to express certain algorithms in an
imperative style.  Consider a function for computing the $n$ first
Fibonacci numbers:

\begin{colorcode}
fun [int] fib(int n) =
  // Create "empty" array.
  let arr = iota(n) in
  // Fill array with Fibonacci numbers.
  loop (arr) = for i < n-2 do
                 let arr[i+2] = arr[i] + arr[i+1]
                 in arr
  in arr
\end{colorcode}

If the array \texttt{arr} is copied for each iteration of the loop, we
are going to put enormous pressure on memory, and spend a lot of time
moving around data, even though it is clear in this case that the
``old'' value of \texttt{arr} will never be used again.  Precisely,
what should be an algorithm with complexity $O(n)$ becomes $O(n^2)$,
due to copying the size $n$ array (an $O(n)$ operation) for each of
the $n$ iterations of the loop.

To prevent this, we will want to update the array \textit{in-place},
that is, with a static guarantee that the operation will not require
any additional memory allocation, such as copying the array.  With an
in-place modification, a \texttt{let-with} can modify the array in
time proportional to the slice being updated ($O(1)$ in the case of
the Fibonacci function), rather than time proportional to the size of
the final array, as would the case if we perform a copy.  In order to
perform the update without violating referential transparency, we need
to know that no other references to the array exists, or at least that
such references will not be used on any execution path following the
in-place update.

In \LO{}, this is done through a type system feature called
\textit{uniqueness types}, similar to, although simpler, than the
uniqueness types of Clean.  Alongside a (relatively) simple aliasing
analysis in the type checker, this is sufficient to determine at
compile time whether an in-place modification is safe, and signal a
compile time error if \texttt{let-with} is used in way where safety
cannot be guaranteed.  This means that \texttt{let-with} must
\textit{always} be efficient, and its use is not permitted otherwise.

The simplest way to introduce uniqueness types is through examples.
To that end, let us consider the following function definition.

\begin{colorcode}
fun \emp{*}[int] modify(\emp{*}[int] a, int i, int x) =
  let b = a with [i] <- a[i] + x in
  b
\end{colorcode}

The function call \texttt{modify($a$,$i$,$x$)} returns $a$, but where
the element at index \texttt{i} has been increased by $x$.  Note the
\emp{asterisks}: in the parameter declaration \texttt{*[int] a}, this
means that the function \texttt{modify} has been given ``ownership''
of the array $a$, meaning that any caller of \texttt{modify} will
never reference array $a$ after the call again.  In particular,
\texttt{modify} can change the element at index \texttt{i} without
first copying the array, i.e. \texttt{modify} is free to do an
in-place modification.  Furthermore, the return value of
\texttt{modify} is also unique - this means that the result of the
call to \texttt{modify} does not share elements with any other visible
variables.

Let us consider a call to \texttt{modify}, which might look as
follows.

\begin{colorcode}
let \(b\) = modify(\(a\), \(i\), \(x\)) in
...
\end{colorcode}

Under which circumstances is this call valid?  Two things must hold:
\begin{enumerate}
\item The type of \texttt{$a$} must be \texttt{*[int]}, of course.

\item Neither \texttt{$a$} or any variable that \textit{aliases}
  \texttt{$a$} may be used on any execution path following the call to
  \texttt{modify}.
\end{enumerate}

In general, when a value is passed as a unique-typed argument in a
function call, we consider that value to be \textit{consumed}, and
neither it nor any of its aliases can be used again.  Otherwise, we
would break the contract that gives the function liberty to manipulate
the argument however it wants.  Note that it is the type in the
argument declaration that must be unique - it is permissible to pass a
unique-typed variable as a non-unique argument (that is, a unique type
is a subtype of the corresponding nonunique type).

A variable $v$ aliases $a$ if they may share some elements,
i.e. overlap in memory.  As the most trivial case, after evaluating
the binding \texttt{let b = a}, the variable \texttt{b} will alias
\texttt{$a$}.  As another example, if we extract a row from a
two-dimensional array, the row will alias its source:
\begin{colorcode}
let b = a[0] in
... // b is aliased to a (assuming a is not one-dimensional)
\end{colorcode}
\Cref{sec:l0-sharing} will cover sharing and sharing analysis in
greater detail.

Let us consider the definition of a function returning a unique array:

\begin{colorcode}
fun *[int] f([int] a) = \(e\)
\end{colorcode}

Note that the argument, \texttt{$a$}, is non-unique, and hence we
cannot modify it.  There is another restriction as well: \texttt{$a$}
must not be aliased to our return value, as the uniqueness contract
requires us to ensure that there are no other references to the unique
return value.  This requirement would be violated if we permitted the
return value in a unique-returning function to alias its (non-unique)
parameters.

To summarise: \textit{values are consumed by being the source in a
  \texttt{let-with}, or by being passed as a \textit{unique} parameter
  in a function call}.  We can crystallise valid usage in the form of
three principal rules:

\begin{description}
\item[Uniqueness Rule 1] When a value is passed in the place of a
  unique parameter in a function call, or used as the source in a
  \texttt{let-with} expression, neither that value, nor any value that
  aliases it, may be used on any execution path following the function
  call.  A violation of this rule is illustrated on
  \cref{fig:uniqueness-rule-1-violation}.

\item[Uniqueness Rule 2] If a function definition is declared to
  return a unique value, the return value (that is, the result of the
  body of the function) must not share memory with any non-unique
  arguments to the function.  As a consequence, at the time of
  execution, the result of a call to the function is the only
  reference to that value.  A violation of this rule is illustrated on
  \cref{fig:uniqueness-rule-2-violation}.

\item[Uniqueness Rule 3] If a function call yields a unique return
  value, the caller has exclusive access to that value.  At
  \textit{the point the call returns}, the return value may not share
  memory with any variable used in any execution path following the
  function call.  This rule is particularly subtle, but can be
  considered a rephrasing of Uniqueness Rule 2 from the ``calling
  side''.
\end{description}

\begin{figure}
\centering
\begin{colorcode}
let b = a with [i] <- 2 in
f(b,a) // \emp{Error:} a used after being source in a let-with
\end{colorcode}
\caption{Violation of Uniqueness Rule 1}
\label{fig:uniqueness-rule-1-violation}
\end{figure}

\begin{figure}
\centering
\begin{colorcode}
fun *[int] broken([[int]] a, int i) =
  a[i] // Return value aliased with 'a'.
\end{colorcode}
\caption{Violation of Uniqueness Rule 2}
\label{fig:uniqueness-rule-2-violation}
\end{figure}

Finally, it is worth emphasising that everything in this chapter is
used as part of a static analysis.  \textit{All} violations of the
uniqueness rules will be discovered at compile time (in fact, during
type-checking), thus leaving the code generator and runtime system at
liberty to exploit them for low-level optimisation.

\subsection{Sharing analysis}
\label{sec:l0-sharing}

Whenever the memory regions for two values overlap, we say that they
are \textit{aliased}, or that \textit{sharing} is present.  As an
example, if you have a two-dimensional array \texttt{a} and extract
its first row as the one-dimensional array \texttt{b}, we say that
\texttt{a} and \texttt{b} are aliased.  While the \LO{} compiler may
do a deep copy if it wishes\footnote{At some point, a proper cost
  model for \LO{} will be developed, and it is very likely that we
  require such indexing to be $O(1)$.}, it is not required, and this
operation thus holds the potential for sharing memory.  Sharing
analysis is necessarily conservative, and merely imposes an upper
bound on the amount of sharing happening at runtime.  The sharing
analysis in \LO{} has been carefully designed to make the bound as
tight as possible, but still easily computable.

In \LO{}, the only values that can have any sharing are arrays -
everything else is considered ``primitive''.  Tuples are special, in
that they are not considered to have any identity beyond their
elements.  Therefore, when we store sharing information for a
tuple-typed expression, we do it for each of its element types, rather
than the tuple value as a whole.

To be precise, sharing information for an expression $e$, written
$\aliases{e}$, can take one of two forms:

\begin{enumerate}
\item $l$, where $l$ is a subset of the variables in scope at $e$.
  This means that $e$ may share data with some of the variables in
  $l$.  This is the sharing information when the type of $e$ is not a
  tuple.

\item $\langle l_{1}, \ldots, l_{n} \rangle$, which requires that the
  type of $e$ is a tuple $\{t_{1}, \ldots, t_{n}\}$, and denotes that
  the sharing of the $i$th component is $l_{i}$.  This is the shape of
  the sharing information when the type of $e$ is a tuple.
\end{enumerate}

We need a way to combine sharing information.  The typical case is
computing sharing information for the expression \texttt{if c then e1
  else e2}, where the sharing of the resulting value is the
``combination'' of the sharing in both \texttt{e1} and \texttt{e2}.
We make this combination precise by the associative, commutative
operation $s_{1} \oplus s_{2}$, which is defined by the following
equation.

\begin{align*}
  l_{1} \oplus l_{2} &= l_{1} \cup l_{2} \\
  \langle l_{1}, \ldots, l_{n} \rangle \oplus \langle l_{n+1}, \ldots, l_{2n} \rangle &= \langle l_{1} \oplus l_{n+1}, \ldots, l_{n} \oplus l_{2n} \rangle \\
\end{align*}

Now we can define
\[
\aliases{\texttt{if c then e1 else e2}} = \aliases{\texttt{e1}} \oplus \aliases{\texttt{e2}}.
\]

We will often treat sharing information as a set and write things such
as $\forall v\in\aliases{e}.p(v)$ -- in these cases, the set elements
are all variables contained anywhere in the sharing information.

Aliasing is transitive -- if $v\in\aliases{e}$ and $v'\in\aliases{e}$,
then $v\in\aliases{v'}$.  Aliasing is mostly intuitive - during
type-checking, the symbol table contains not only the type of each
variable, but also which other variables it may alias.  Hence, we can
define an aliasing rule for variables:

\begin{align*}
  \aliases{\texttt{$v$}} &= \{v\} \cup \{\textrm{Any variable in scope that aliases \(v\)}\} \\
\end{align*}

The aliasing rules for other expressions are mostly intuitive, but a
few interesting cases are presented here:

\begin{align*}
  \aliases{\texttt{$e$}} &= \emptyset\ (\text{Whenever $e$ has a basic type\footnote{A basic type is a type that is not contain an array anywhere.}}) \\
  \aliases{\texttt{$a$[i]}} &= \aliases{a} \\
  \aliases{\texttt{copy($e$)}} &= \emptyset \\
  \aliases{\texttt{if $c$ then ${e}_{1}$ else ${e}_{2}$}} &= \aliases{e_{1}} \oplus \aliases{e_{2}} \\
  \aliases{\texttt{transpose($e$)}} &= \aliases(e) \\
\end{align*}

Note that \texttt{transpose} introduces aliasing - this is to permit
an implementation where the transposed array is never actually
manifested in memory, but is merely an index space transformation of
the underlying array resolved at compile-time.  The operations
\texttt{reshape}, \texttt{split}, etc. have a similar rule.

The rule for function application is more complicated.  To begin with,
and this was indeed the original rule in \LO{}, we can state that the
return value of a function call aliases all of its arguments.

\begin{align*}
  \aliases{\textit{f}(e_{1}, \ldots, e_{n})} &= \bigcup_{1 \leq i \leq n} \aliases{e_{i}} & \textit{(--Too restrictive!)}
\end{align*}

However, it turns out that this is far too restrictive.  Consider a
call \texttt{f1($a$)} to the function \texttt{f1} whose type is shown
on \cref{fig:unique-arguments} - if the return value aliased the
argument $a$, then we could never use the return value at all, as it
would alias something that has been consumed, namely the parameter
$a$:

\begin{colorcode}
let x = f1(a) in // Now 'x' would alias 'a'.
x                // Violates Uniqueness Rule 1,
                 // as something aliasing 'a' is accessed
\end{colorcode}

Hence, a first elaboration is that the return value should only alias
those function arguments that are not consumed:

\begin{align*}
  \aliases{\textit{f}(e_{1}, \ldots, e_{n})} &= \bigcup_{1 \leq i \leq n, \text{$e_{i}$ is not consumed}} \aliases{e_{i}}  & \textit{(--Still too restrictive!)}
\end{align*}

The argument for the soundness of this rule is as follows: even if the
return value may at runtime alias a consumed argument, we do not need
to record it, as that argument will never be accessed elsewhere.

Unfortunately, the above rule is still too restrictive, as can be
illustrated by function \texttt{f2} from \cref{fig:unique-arguments}.
Consider a call \texttt{f2($a$)} - by the above rule, the return value
would be aliased to $a$, which would violate Uniqueness Rule 3, as $a$
may be used again.

Hence, we add another elaboration, wherein the alias set is empty if
the return value is unique.

\begin{align*}
  \aliases{f(e_{1}, \ldots, e_{n})} &=
  \begin{cases}
    \emptyset & \mbox{If $f$ returns an unique value}\\
    \bigcup_{\overset{1 \leq i \leq n}{\text{$e_{i}$ is not consumed}}} \aliases{e_{i}} & \mbox{Otherwise}
  \end{cases}
\end{align*}

The final rule is essentially correct, except that it ignores tuples.
As mentioned earlier, sharing information for tuples is represented
element-wise.  Hence, we can simply apply the above rule piecewise for
each element in the tuple.

\begin{figure}
\begin{center}
\begin{colorcode}
fun [int] f1(*[int] a) = ...

fun *[int] f2([int] a) = ...
\end{colorcode}
\end{center}
\caption{Unique arguments}
\label{fig:unique-arguments}
\end{figure}

Although the current aliasing rules for function calls have proven
sufficient for now, there are cases where it is too conservative.
Consider the following function.

\begin{colorcode}
fun [int] contrived([[int]] src, [int] indexes, int i) =
  src[indexes[i]]
\end{colorcode}

In a call \texttt{contrived(src,indexes,i)}, by the above rules, we
would consider the return value to be aliased to \textit{both}
\texttt{src} and \texttt{indexes}, as both are non-consumed
parameters.  Yet, it is clear by inspecting the actual function
definition that the return value will only index the \texttt{src}
parameter.

This problem is not solvable merely through refinement of the aliasing
rules - either the user must annotate each function with information
about which of the parameters may be aliased by the return value, or
the compiler could deduce it using some sharing inference algorithm.
As the latter would add a great deal of complexity, and the former
require a language change, we have postponed tackling this problem
until it becomes a problem in practice.

\subsection{Tracking uniqueness}
\label{sec:l0-tracking-uniqueness}

Let us summarise:

\begin{itemize}
\item If the type of an array parameter is preceded by a single
  asterisk, it denotes that the array is unique, i.e., that it will
  never be reused outside of the current function.

\item The source operand to a \texttt{let-with} \textit{must} be
  unique.  If it is not, it is reported as a type error.
\end{itemize}

Let-with and function calls are the only places in which variable
consumption can happen.  As a first example, let us consider a
function that replaces the value at a given position in an integer
array.

\begin{colorcode}
  fun *[int] replace(*[int] arr, int i, int x) =
    let arr[i] = x in arr
\end{colorcode}

The type of this function expresses the fact that it consumes its
array argument, and also returns a unique array.  This permits
composition - \texttt{replace(replace(a, i1, x), i2, y)} is a valid
application.  Defining \texttt{replace} as
\begin{colorcode}
  fun [int] replace2(*[int] arr, int i, int x) =
    let arr[i] = x in arr
\end{colorcode}
would still be type correct (a unique array can be used anywhere a
nonunique is expected), but the composition
\texttt{replace2(replace2(a, i1, x), i2, y)} would no longer be well
typed.

Checking that uniqueness invariants are being upheld is far subtler
than normal type checking.  In particular, detailed sharing analysis
has to be performed, in order to ensure that after an array $a$ is
consumed, it becomes an error to use any value that may refer to
(parts of) the old value of the array.  Whenever we consume a variable
$a$, we mark as inaccessible all of its aliases, as illustrated on
\cref{fig:consumption}.

\begin{figure}
\begin{center}
\begin{colorcode}
  let b = a in               // Now \(\texttt{b}\in\aliases{\texttt{a}}\).
  let c = a with [i] <- x in // \(\forall{}v\in\aliases{\texttt{a}}\Rightarrow\textrm{Mark \(v\) as consumed.}\)
  b                          // Error, because \(\texttt{b}\in\aliases{\texttt{a}}\)!
\end{colorcode}
\end{center}
\caption{Example of array consumption}
\label{fig:consumption}
\end{figure}

A key principle is that of \textit{sequence points} that lexically
checkpoint the use of variables.  As an example, assume that we are
given a function \texttt{f} of type \texttt{*[int]~->~int}.  That is,
\texttt{f} consumes an array and returns an integer.  The expression
\begin{colorcode}
  f(a) + a[i]
\end{colorcode}
is invalid because a consumption and observation of the same variable
happens within the same \textit{sequence}.  It is valid for a sequence
to contain multiple observations of the same variable, but if a
variable is consumed, that must be the only occurrence of the variable
(or any of its aliases) within the sequence.  Binding constructs
(lets, let-withs and loops) create sequence points that delimit
sequences.  If we rewrite the expression to coordinate the consumption
into its own sequence, all will be well.
\begin{colorcode}
  let c = a[i] // Since a[i] is of primitive type,
               // c does not alias a.
  in f(a) + c
\end{colorcode}

The reason for this rule is to enable simpler code generation, as any
necessary order of operations is evident in the code.  It does require
a certain amount of care when doing program transformations, as for
example expression reordering may result in an invalid program, as
shown on \cref{fig:reordering-uniqueness-violation}.

\begin{figure}
\centering
\begin{minipage}{0.25\columnwidth}
\begin{colorcode}
let x = a[0] in
let b = a with
  [i] <- y in
x + b[1]
\end{colorcode}
\end{minipage}
\begin{minipage}{0.05\columnwidth}
$\Rightarrow$
\end{minipage}
\begin{minipage}{0.6\columnwidth}
\begin{colorcode}
let b = a with [i] <- y in
let x = a[0] in // \emp{Error}:
                // violates Uniqueness Rule 1
x + b[1]
\end{colorcode}
\end{minipage}
\caption{Expression reordering causing violation of uniqueness rules}
\label{fig:reordering-uniqueness-violation}
\end{figure}

In the previous examples, function arguments that were consumed were
all simple variables, making it easy to describe what was being
consumed.  But in general, we might have an expression
\begin{colorcode}
  replace(e, i, x)
\end{colorcode}
where \texttt{e} is some arbitrary expression.  In this case, we mark
as consumed all variables in $\aliases{\texttt{e}}$.

Constant, literal arrays are not considered unique, as the compiler
may put them in read-only memory and return the same reference every
time they are accessed.  For example, the following program is
invalid.
\begin{colorcode}
  fun [int] fibs(int i, int x) =
    let a = [1, 1, 2, 3, 5, 8, 13] in
    let a[i] = x in a
\end{colorcode}
Since \texttt{a} is not unique, its use in the let-with is a type
error.  However, we can use \texttt{copy} to create a unique duplicate
of the array.
\begin{colorcode}
  fun [int] fibs(int i, int x) =
    let a = copy([1, 1, 2, 3, 5, 8, 13]) in
    let a[i] = x in a
\end{colorcode}

If we have a function such as
\begin{colorcode}
  fun int f(*[int] a, int x) = x
\end{colorcode}
then it is not valid to curry it in such a way that we provide values
for the consumed parameters.  For example, \texttt{map(f ($a$), b)}
would be an error.  The reason for this is that \texttt{f} may be
called an arbitrary number of times during the mapping, but $a$ can
only be consumed once.

\end{document}
